\documentclass[bigger]{beamer}

\usepackage{amssymb}
\usepackage{amsfonts}
\usepackage{amsmath}
\usepackage{mathpazo}

 \let\TEXTsymbol\ensuremath 
 
\usetheme{Madrid}
\usecolortheme{beaver}
\usefonttheme{professionalfonts}

\begin{document}

\title[47-805: Intro]{47-805: Computational Methods\\
Introduction \& General issues}
\subtitle{Judd Chapters 1-2}
\author[David Childers]{David Childers (thanks to Y. Kryukov, K. Judd, and U. Doraszelski)}
\institute[CMU]{CMU, Tepper School of Business}
\date[Mar-13]{March 13, 2023}
\maketitle

\begin{frame}%

\frametitle{Plan for today}

\begin{enumerate}
\item Computational methods: why and how to use them

\item General points on:

\begin{enumerate}
\item Number representation

\item Problem solving

\item Coding\bigskip
\end{enumerate}
\end{enumerate}

Next time: Solving systems of linear equations (Chapter 3)


\end{frame}%

\begin{frame}%

\frametitle{What are computational methods?}

\begin{itemize}
\item Economics is full of problems without closed-form solutions:

\begin{itemize}
\item Macro: dynamic problems

\item Econometrics: estimators solve optimization or integration problems

\item General equilibrium is a system of nonlinear eq-s

\item Finance: stochastic calculus \& diff. eq's

\item Applied Micro: Games, especially dynamic ones
\end{itemize}

\item Pencil \& paper methods limit us to special cases

\item Realistic applications lead to analytically intractable problems

\item Numerical methods let us substantially relax constraint on tractability
\end{itemize}


\end{frame}%

\begin{frame}%

\frametitle{Computing power \& its limits}

\begin{itemize}
\item Moore's law: computing power doubles every 18 months

\begin{itemize}
\item Faster CPUs, more CPUs in the same box

\item Server farms / Supercomputers: 100's of CPUs/GPUs
\end{itemize}

\item But: CPU is a complement to human brain, not a substitute

\begin{itemize}
\item Brute force can take too long, or fail to solve the problem.\newline
$\Rightarrow $ choice of model form, solution method, starting value, etc.

\item Inappropriate methods can produce unreasonable results\newline
$\Rightarrow $ economic interpretation \& verification of results.
\end{itemize}

\item Numerical methods is a field dedicated to solving problems

\begin{itemize}
\item Economics is about finding interesting problem to solve, \newline
and interpreting the solution; \textbf{not} the solution methods

\item But: need to know strength and weaknesses of each method
\end{itemize}
\end{itemize}


\end{frame}%

\begin{frame}%

\frametitle{What can numerical methods do?}

\begin{itemize}
\item Solve intractable problems:

\begin{itemize}
\item Realistic models in both macro and micro

\item Specialized estimation problems (GMM, MLE, Bayes)

\item Policy experiments on estimated model
\end{itemize}

\item Generate hypothesis or counterexamples

\begin{itemize}
\item Can say: X might happen, maybe even likely/unlikely

\item Cannot say: X will always/never happen
\end{itemize}

\item Check derivations, quantify \& visualize effects

\item Still, can only solve one parameterization at a time:

\begin{itemize}
\item Large number of examples instead of general results
\end{itemize}
\end{itemize}


\end{frame}%

\begin{frame}%

\frametitle{Criticism of numerical methods, \& response}

\begin{itemize}
\item Limited to a finite number of parameterizations

\begin{itemize}
\item Theory is usually limited to a finite set of functional forms

\item We can approximate smooth functions

\item Can combine w/ symbolic/logical methods for non-local results  
\end{itemize}

\item Introduces errors

\begin{itemize}
\item We can bound or measure them: compute and verify
\end{itemize}

\item Is a "Black box": what are economic forces behind the result?

\begin{itemize}
\item Do comparative statics, e.g. try turning effects off

\item Take analytical derivation as far as you can
\end{itemize}

\item Computational methods extend analytical ones, not replace them
\end{itemize}

\end{frame}%

\begin{frame}%

\frametitle{Computational research stages}

\begin{enumerate}
\item Pick an economic issue, represent as mathematical problem

\item Pick a numerical solution method

\item Code it up, solve, experiment with it

\item Interpret the results:

\begin{enumerate}
\item Note behavior consistent with existing theory

\item Find economic reasons behind unexpected behavior,\newline
make sure it is not caused by numeric issues

\item Solve trivial cases (e.g. discount factor = 0), \newline
where you know how the solution should look like

\item Robustness checks: different numeric method(s) \& settings, \newline
different functional forms, different model
\end{enumerate}

\item Use data to quantify model features, test structure 

\item 4.1-2 or 5 core of theory, empirical paper, respectively, \newline
4.4 = brief section at the end + details in online appendix
\end{enumerate}

\end{frame}

\begin{frame}%

\frametitle{Course goals}

\begin{itemize}
\item Solve dynamic problems (e.g. discrete time):%
\begin{eqnarray*}
&&\max \sum \beta ^{t}u\left( f\left( k_{t}\right) -k_{t+1}\right) \\
&\Rightarrow &V\left( k\right) =\max_{k^{\prime }}u\left( f\left( k\right)
-k^{\prime }\right) +\beta V\left( k^{\prime }\right)
\end{eqnarray*}

\item Maximization $\Rightarrow $ nonlinear equations $\Rightarrow $ linear
equations

\begin{itemize}
\item Also: constraints, integration ($\beta \mathbf{E}\left[ V\left(
k^{\prime }\right) |k,c\right] $), etc.
\end{itemize}

\item Unknown function $V\left( \cdot \right) $: polynomials or splines

\item Solving the dynamic problem:

\begin{itemize}
\item Dynamic programming (iterate on Bellman)

\item Projection: find $V\left( \cdot \right) $ and policy functions directly
\end{itemize}
\end{itemize}

\end{frame}

\begin{frame}

\frametitle{Model taxonomy}

\begin{itemize}
\item Time:

\begin{itemize}
\item Discrete: $V\left( k\right) =\max_{c}u\left( k,c\right) +\beta V\left(
f(k,c)\right) $

\item Continuous: $\ \rho V\left( k\right) =\max_{c}u(k,c)+V^{\prime }\left(
k\right) g(k,c)$
\end{itemize}

\item State:

\begin{itemize}
\item Discrete ($k\in \left\{ k_{i}\right\} $)

\item Continuous ($k\geq 0$)
\end{itemize}

\item State transitions:

\begin{itemize}
\item Deterministic: ... $+\beta V\left( f(k,c)\right) $

\item Stochastic: ... $+\beta \mathbf{E}_{k^{\prime }}\left[ V\left(
k^{\prime }\right) |k,c\right] $
\end{itemize}

\item We can solve any combination of the above

\begin{itemize}
\item Pick features that match your economic story

\item "Simplest" case (discrete time \& state, deterministic transitions)
often leads to unreasonable results, \newline
and can be hard to compute.
\end{itemize}
\end{itemize}

\end{frame}%

\begin{frame}%

\frametitle{Course structure}

\begin{enumerate}
\item Solving for an unknown number (or vector):

\begin{itemize}
\item Solving linear and nonlinear equations

\item Optimization

\item Integration and simulation
\end{itemize}

\item Solving for an unknown function:

\begin{itemize}
\item Polynomial, spline, \& other approximations

\item Differential equations

\item Projection methods
\end{itemize}

\item Applications to dynamic problems in Econ:

\begin{itemize}
\item Dynamic programming

\item Optimal control, Euler equations

\item Other advanced topics (suggestions?)
\end{itemize}
\end{enumerate}


\end{frame}%

\begin{frame}%

\frametitle{Tradeoffs}

\begin{itemize}
\item Time vs. precision

\item Time vs. robustness

\item Machine time vs. programming time\bigskip
\end{itemize}


\end{frame}%

\begin{frame}%
%EndExpansion
\frametitle{Course logistics}

\begin{itemize}
\item Canvas: lectures, homeworks, announcements

\item \textbf{Homeworks} - your best chance to review the material, learn to
code \& receive feedback

\begin{itemize}
\item Submit meaningful write-up \& discussion; upload code as ZIP

\item Teams up to 3 people; HW 1 should be done alone.

\item Can't use outside help, e.g.: other students, last year solutions, 
\newline
code from other sources

\item Can use Internet for reference purposes (with citation)

\item Lowest scoring homework will be dropped (given honest effort)
\end{itemize}

\item No midterm

\item Final exam: 48 hour take-home, date TBD
\end{itemize}

\end{frame}


\begin{frame}%

\frametitle{Number representation \& machine zero}

\begin{itemize}
\item At lowest (binary) level, computer can only deal with integers

\item Real numbers are stored as $\pm m\ast 2^{\pm n}$

\begin{itemize}
\item $m$ = mantissa, $n$ = exponent
\end{itemize}

\item \textbf{Double precision} storage format: $m$ and $n$ share the same
64 bits $\implies $ both are bounded

\item Bound on exponent $n$ determines lowest and largest values

\item \textbf{Machine zero}: smallest positive number that machine can
represent

\begin{itemize}
\item Matlab: \texttt{realmin}/Julia \texttt{nextfloat(-Inf)}/Numpy \texttt{np.finfo(np.float64).min} $\approx $ -2e-308

\item Largest possible number: \texttt{realmax}/\texttt{prevfloat(Inf)}/\texttt{np.finfo(np.float64).max} $\approx $ \texttt{2e+308 }$%
\approx e^{710}$
\end{itemize}
\end{itemize}


\end{frame}%

\begin{frame}%

\frametitle{Machine epsilon}

\begin{itemize}
\item Real numbers are stored as $\pm m\ast 2^{\pm n}$
\end{itemize}

Bound on mantissa $m$ limits the number of digits:

\begin{itemize}
\item \textbf{Machine epsilon}: smallest difference from 1.0\newline
that the machine can represent

\item Given Matlab/Julia/Python 64-bit precision: \texttt{eps()} $\approx $ 2e-16

\begin{itemize}
\item If you have $A\approx 1$, and \texttt{A-B }$\approx 1e$-$16$, \newline
then you can assume that $A=B$.

\item Example: type "\texttt{0.6-0.2-0.2-0.2}" into Matlab/Julia/Python
\end{itemize}

\item Machine zero \TEXTsymbol{<} machine epsilon:

\begin{itemize}
\item $m$ and $n$ share the same 64 bits

\item many digits in $m$ leave less room for $n$

\item change units to keep your values in (0.01, 100)
\end{itemize}
\end{itemize}

\end{frame}%

\begin{frame}%
%EndExpansion
\frametitle{Error propagation}

\begin{itemize}
\item Mathematical operations propagate rounding error

\item Solving: $x^{2}-26x+1=0\implies x=13-\sqrt{168}=0.0385186$

\item Machine that stores 5 decimal digits would compute: $%
13.000-12.961=0.039$

\item Answer has precision reduced from 5 meaningful digits to 2

\begin{itemize}
\item Small difference between large numbers will be imprecise
\end{itemize}

\item If $A$ and $B$\texttt{\ }are on the scale of $10^{6}$, \newline
you cannot expect \texttt{A-B} to go below \texttt{1e-10}
\end{itemize}


\end{frame}%

\begin{frame}%
%EndExpansion
\frametitle{Computation speed}

\begin{itemize}
\item Basic arithmetic (+, --, *, /) is fast

\item Everything else is approximated through those

\item Functions with no closed form take a lot longer

\item Exponent takes several times longer than multiplication:%
\begin{equation*}
e^{x}=\sum\nolimits_{n=0}^{\infty }\frac{x^{n}}{n!}\approx
\sum\nolimits_{n=0}^{N}\frac{x^{n}}{n!}
\end{equation*}

\item $\Rightarrow $ pre-compute \& store expressions that will not change
during iterations

\item $\Rightarrow $ transform the formula, e.g. Horner's method:

\begin{itemize}
\item $a_{0}+a_{1}x+a_{2}x^{2}+a_{3}x^{3}$ , vs.

\item $a_{0}+x\left( a_{1}+x\left( a_{2}+a_{3}x\right) \right) $
\end{itemize}
\end{itemize}


\end{frame}%

\begin{frame}%

\frametitle{Solution methods}

\textbf{Direct methods} (e.g. $x=\frac{-b\pm \sqrt{b^{2}-4ac}}{2a}$):

\begin{itemize}
\item Result in an "exact" solution

\item Takes finite (but potentially long) time
\end{itemize}

\textbf{Iterative methods}:

\begin{itemize}
\item Each iteration $k$ generates a new guess at the solution:\newline
$x_{k+1}=g\left( x_{k},x_{k-1},...\right) $.

\item E.g. $x_{k}=\left[ e^{a}\right] _{k}=\sum\nolimits_{n=0}^{k}\frac{a^{n}%
}{n!}=x_{k-1}+\frac{a^{k}}{k!}$

\item Each guess is (hopefully) closer to the true solution

\begin{itemize}
\item I.e. sequence (hopefully) converges, to the solution
\end{itemize}

\item Stopping criterion \& max. number of iterations trade off computation
time vs. precision.
\end{itemize}


\end{frame}%

\begin{frame}
\frametitle{Order notation}

\begin{itemize}

\item Let $x_k$ be the $k^{th}$ element of a sequence and $f(k)$ a nonnegative sequence 

\item $x_{k}$ is $O(f(k))$ (say "$x_k$ is big O $f(k)$") if

\begin{equation*}
\underset{k\to\infty}{\lim\sup} \left|\frac{x_{k}}{f(k)}\right|<\infty
\end{equation*}

\item $x_{k}$ is $o(f(k))$ (say "$x_k$ is little o $f(k)$") if

\begin{equation*}
\underset{k\to\infty}{\lim}\left|\frac{x_{k}}{f(k)}\right|= 0 
\end{equation*}


\end{itemize}

\end{frame}

\begin{frame}%

\frametitle{Computational Complexity: Some minimal theory}

\begin{itemize}

\item Goal of computation is to take some (finite) input $x$ (a \emph{problem}) and produce the correct output $y$ (the "result") by means of a function $f(.)$ (a \emph{program}) that applies a series of steps which can be performed by a machine
\item Standard to study case where $y\in\{True,False\}$
\item Computational theory asks several questions about $f()$

\begin{enumerate}
\item Computability: $\exists f()$ which produces $y$ in finite time?
  \begin{itemize}
  \item Fact: $\exists$ problems for which answer is \emph{no}:
  \item These problems are called "noncomputable"
  \end{itemize}
\item Supposing answer to (1) is *yes*, how many resources does it take to compute?
  \begin{itemize}
  \item Time: $\#$ of operations
  \item Space: $\#$ of bits of memory
  \item Other resources as needed
  \end{itemize}
\end{enumerate}

\end{itemize}

\end{frame}%

\begin{frame}%

\frametitle{Computational Resources}

\begin{itemize}

\item Answers to resource cost allowed can depend on size $n$ of input
    \begin{itemize}
    \item If $\#$ ops (bits) $O(n^k)$, $k<\infty$ problem is polynomial time (space)
    \item If $\#$ ops (bits) $O(2^{n^k})$ problem is exponential time (space)
    \end{itemize}
\item Biggest unsolved problem in compsci is, roughly, what kinds of problems are in each class
\item Numerics looks at problems where $y$ is a number, which allows consideration of other resources
    \begin{itemize}
    \item Precision: For $\epsilon>0$, let $f(x)$ produce $\tilde{y}$ such that $\left\Vert y-\tilde{y}\right\Vert<\epsilon$
    \item Probability: For $\delta>0$ allow $\tilde{y}$ random such that $Pr(\left\Vert y-\tilde{y}\right\Vert<\epsilon)>1-\delta$
    \end{itemize}


\end{itemize}

\end{frame}%

\begin{frame}%

\frametitle{Approach to resources in this class}

\begin{itemize}


\item Goal: find programs for problems $x$ in some class which are fast, small, precise, reliable.
    \begin{itemize}
    \item i.e. $ops<O(q(\epsilon,\delta,n))$ with optimal dependency of $q()$ on $\epsilon,\delta,n$
    \item Equivalently, $\epsilon<O(r(ops,\delta,n))$ for optimal $r()$
    \end{itemize}

\item We will not systematically probe optimal production possibilities frontier for any class of problem
\item Instead, we will describe reasonable assortment of useful methods, and say a little bit about properties and tradeoffs


\end{itemize}

\end{frame}

\begin{frame}%

\frametitle{Rates of convergence (geometric convention)}

\begin{itemize}
\item $x_{k}$ converges to $x^{\ast }$ at rate $q$ if:%
\begin{equation*}
\lim_{k\rightarrow \infty }\frac{\left\Vert x_{k+1}-x^{\ast }\right\Vert }{%
\left\Vert x_{k}-x^{\ast }\right\Vert ^{q}}<\infty
\end{equation*}

\item $x_{k}$ converges to $x^{\ast }$ \textbf{linearly} at rate $\beta $ if:%
\begin{equation*}
\lim_{k\rightarrow \infty }\frac{\left\Vert x_{k+1}-x^{\ast }\right\Vert }{%
\left\Vert x_{k}-x^{\ast }\right\Vert }=\beta <1
\end{equation*}

\item Convergence at rate $q$ $\implies$ $\left\Vert x_{k}-x^{\ast}\right\Vert=O(\exp(-q^{k}))$

\item Linear convergence at rate $\beta$ $\implies$ $\left\Vert x_{k}-x^{\ast}\right\Vert=O(\beta^{k})$

\item Informally: Linear convergence improves precision by 1 digit each fixed $\#$ of iterations ($\frac{-1}{\log_{10}(\beta)}$)

\item Iterative algorithms usually have geometric convergence

\item Will use this convention in most of class

\end{itemize}


\end{frame}%

\begin{frame}%
%EndExpansion
\frametitle{Stopping rules}

\begin{itemize}
\item Want to stop when $x_{k}$ is close to $x^{\ast }$.

\item We do not know $x^{\ast }$ $\Rightarrow $ stop when $x_{k}$ is close
to $x_{k+1}$, \newline
i.e. when $f\left( \left\Vert x_{k+1}-x_{k}\right\Vert \right) <\delta $

\item Using absolute difference $\left\Vert x_{k+1}-x_{k}\right\Vert $ is a
bad idea \newline
if $x$'s are large, due to limited precision.

\item Relative difference $\left\Vert x_{k+1}-x_{k}\right\Vert /\left\Vert
x_{k}\right\Vert $ -- better but \newline
might "blow up" if $x_{k}$ is close to zero

\item Hybrid rule: $\left\Vert x_{k+1}-x_{k}\right\Vert /\left[ 1+\left\Vert
x_{k}\right\Vert \right] $

\item Potential problem: we might stop far from $x^{\ast }$

\begin{itemize}
\item E.g. $x_{k}=\sum\nolimits_{n=1}^{k}\frac{1}{n}\rightarrow \infty $,
but $\left\vert x_{k}-x_{k+1}\right\vert =\frac{1}{k+1}\rightarrow 0$
\end{itemize}
\end{itemize}


\end{frame}%

\begin{frame}%

\frametitle{Testing the solution: compute and verify}

\begin{itemize}
\item Solving $f\left( x\right) =0$: true solution is $x^{\ast }$ is
unknown, \newline
computed numeric solution is $\hat{x}$

\item Forward error analysis (compare $x^{\ast }$ to $\hat{x}$) is infeasible

\item Backward error analysis: compare $f\left( \cdot \right) $ to similar
function $\hat{f}\left( \cdot \right) $ that has $\hat{f}\left( \hat{x}%
\right) =0$

\begin{itemize}
\item $\hat{f}\left( \cdot \right) $ can be tricky to make meaningful
\end{itemize}

\item Compute and verify: compare $f\left( \hat{x}\right) $ to $0$, \newline
normalized for scale if possible

\begin{itemize}
\item GE: $E\left( p^{\ast }\right) \equiv D\left( p^{\ast }\right) -S\left(
p^{\ast }\right) =0$ (absolute error)

\item Relative error: compare $E\left( \hat{p}\right) /D\left( \hat{p}%
\right) $ to $0$
\end{itemize}
\end{itemize}


\end{frame}%

\begin{frame}%

\frametitle{Matlab hints}

\begin{itemize}
\item Every variable is a matrix (or $N$-dimensional array),\newline
of double-precision reals.

\item Capitalization matters

\item Interpreter language: check variables after error

\begin{itemize}
\item \emph{F12} = break point; \texttt{error('Message')} stops code
\end{itemize}

\item Use help:

\begin{itemize}
\item \emph{Menu/Help/Product help}, then maybe \emph{Index}

\item Select command and press F1

\item Type: \texttt{help \TEXTsymbol{<}command\TEXTsymbol{>}}
\end{itemize}

\item Matrix operations instead of \texttt{for }loops:

\begin{itemize}
\item Elementwise operations: \texttt{.+ \ .- .* ./ .\symbol{94}}

\item \texttt{Log(), exp()} and most functions work on matrices

\item indexing: \texttt{lag\_X=[NaN X(1:end-1)];}
\end{itemize}
\end{itemize}


\end{frame}

\begin{frame}

\frametitle{Readable code}

\begin{itemize}
\item Header comment $\%\%$ explaining what this code does

\item Set all parameters as variables at the beginning of the code

\item Use descriptive comments $\%$ every few lines

\begin{itemize}
\item \emph{Ctrl-R} (\emph{command /} on Mac) and \emph{Ctrl-T} = (un)comment lines
\end{itemize}

\item Use offsets with control structures (\texttt{for, while, if})

\begin{itemize}
\item \emph{Ctrl-[} and \emph{Ctrl-]} = offset selected lines
\end{itemize}

\item Cell mode (=paragraphs): "\%\%" starts a new cell

\begin{itemize}
\item Ctrl-Enter runs the current cell
\end{itemize}
\end{itemize}

\end{frame}%

\begin{frame}
\frametitle{Writing the code}

\begin{itemize}
\item One does not simply type 500 lines of code

\begin{itemize}
\item Mistakes will happen
\end{itemize}

\item Try to test every "block" of code

\begin{itemize}
\item F9 runs selected code

\item Use Matlab's cells (\%\% Headers, Ctrl-Enter to run)

\item Give it trivial inputs, so you can know the correct output

\item Verify output using manual calculation or alternative formula
\end{itemize}

\item Provide meaningful output on screen

\begin{itemize}
\item E.g. iteration report: $x$, difference from stopping criterion.

\item Look up \texttt{fprintf} for formatted output
\end{itemize}
\end{itemize}

%TCIMACRO{\TeXButton{EndFrame}{\end{frame}}}%
%BeginExpansion
\end{frame}

\begin{frame}%

\frametitle{Julia hints}

\begin{itemize}
\item Every variable has a \emph{type} (Float64, Int64, Array, etc) and every function must be compatible with the type of the input.

\item Programming with types improves speed and readability by specializing method to input structure.

\item Capitalization matters

\item Use help:

\begin{itemize}

\item \url{https://docs.julialang.org/} and \url{https://julia.quantecon.org} invaluable sources

\item Type: \texttt{? \TEXTsymbol{<}command\TEXTsymbol{>}}
\end{itemize}

\item \texttt{for }loops: not slow like Matlab, but \emph{broadcasting} still useful

\begin{itemize}
\item Elementwise operations: \texttt{.+ \ .- .* ./ .\symbol{94}}

\item \texttt{log.(), exp.()} etc elementwise using $.$ symbol, to whole array without.
\end{itemize}
\end{itemize}


\end{frame}

\begin{frame}

\frametitle{Readable code}

\begin{itemize}

\item Code in Jupyter notebooks mixes text cells and code cells
\begin{itemize}
\item Provide useful header plus context and explanations
\end{itemize}

\item Use notebooks for exploratory model building, collections .jl scripts (in terminal or IDE like VSCode) for large projects 

\item Code in scripts should have header and comments 

\item Use descriptive comments $\#$ every few lines

\begin{itemize}
\item \emph{Ctrl-/} (\emph{command /} on Mac) in VSCode IDE
\end{itemize}

\item Set all parameters as variables at the beginning of the code

\item Use offsets with control structures (\texttt{for, while, if})

\end{itemize}

\end{frame}%





\begin{frame}
\frametitle{Writing the code}

\begin{itemize}
\item One does not simply type 500 lines of code

\begin{itemize}
\item Mistakes will happen
\end{itemize}

\item Try to test every "block" of code

\begin{itemize}
\item Run code in REPL (interactively) to see output

\item Encapsulate discrete behaviors in functions: prevents repetition, improves speed, allows repurposing code when model changes

\item \texttt{using Test} package allows setting formal tests

\item Verify output using manual calculation or alternative formula
\end{itemize}

\item Provide meaningful output on screen

\begin{itemize}
\item E.g. iteration report: $x$, difference from stopping criterion.

\item Look up \texttt{println} for formatted output
\end{itemize}
\end{itemize}

\end{frame}

\begin{frame}%

\frametitle{Python hints}


\begin{itemize}

\item Spaces and capitalization matter, indices start at 0 instead of 1

\item Use help:

\begin{itemize}

\item \url{https://python.quantecon.org} invaluable source

\item Colab and VSCode have docs as popup
\end{itemize}

\item Python is \emph{interpreted} and \emph{dynamically typed}: 
\begin{itemize}
\item Commands make a guess at type of data at runtime rather than requiring explicit declaration
\item Good for convenience, interactivity, bad for reliability and speed
\item \emph{Libraries} like \texttt{numpy}, \texttt{scipy} have fast precompiled functions
\item \texttt{numba} allows \texttt{@jit} (just-in-time) compilation to optimize own functions
\end{itemize}


\end{itemize}


\end{frame}

\begin{frame}

\frametitle{Readable code}

\begin{itemize}

\item Code in Jupyter notebooks mixes text cells and code cells
\begin{itemize}
\item Provide useful header plus context and explanations
\end{itemize}

\item Use notebooks for exploratory model building, collections of scripts (in terminal or IDE like VSCode) for large projects 

\item Code in scripts should have header and comments 

\item Use descriptive comments $\#$ every few lines

\begin{itemize}
\item \emph{Ctrl-/} (\emph{command /} on Mac) in VSCode IDE
\end{itemize}

\item Set all parameters as variables at the beginning of the code

\item Use offsets with control structures (\texttt{for, while, if})

\end{itemize}

\end{frame}%





\begin{frame}
\frametitle{Writing the code}

\begin{itemize}
\item One does not simply type 500 lines of code

\begin{itemize}
\item Mistakes will happen
\end{itemize}

\item Try to test every "block" of code

\begin{itemize}
\item Run code in interactively to see output

\item Encapsulate discrete behaviors in functions and classes: prevents repetition, improves speed, allows repurposing code when model changes

\item Verify output using manual calculation or alternative formula
\end{itemize}

\item Provide meaningful output on screen

\begin{itemize}
\item E.g. iteration report: $x$, difference from stopping criterion.
\item \texttt{print(f"Text $\{$variable$\}$")} for formatted output

\end{itemize}
\end{itemize}

\end{frame}




\end{document}

